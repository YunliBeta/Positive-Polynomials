\documentclass[dvipsnames,10pt]{article}
\usepackage{graphicx,float}
\usepackage{amsmath, amsfonts, amsthm, amssymb}
\usepackage{braket, nicefrac}
\usepackage{hyperref,mathtools}
\usepackage{tikz}
\usepackage{tikz-cd}
\usepackage{theoremref}
\usepackage[a4paper, total={5.5in, 9.1in}]{geometry}
\usepackage{setspace}
\usepackage{xcolor}
\usepackage{tgpagella}

\newcommand{\bb}[1]{\mathbb{#1}}
\newcommand{\cc}[1]{\mathcal{#1}}
\newcommand{\f}[1]{\mathfrak{#1}}
\newcommand{\rad}[1]{\sqrt{\mathfrak{#1}}}
\newcommand{\spec}{\mathrm{Spec}}
\newcommand{\clop}{\mathrm{Clop}}
\newcommand{\p}{\mathfrak{p}}
\newcommand{\tr}{\mathrm{tr}}
\newcommand{\q}{\mathfrak{q}}
\newcommand{\m}{\mathfrak{m}}
\newcommand{\oo}{\mathcal{O}}
\newcommand{\Hom}{\mathrm{Hom}}
\newcommand{\w}[1]{\widetilde{#1}}
\newcommand{\vk}{\mathsf{Var}_k}
\newcommand{\im}{\mathrm{image}\,}
\newcommand{\lcm}{\mathrm{lcm}}
\newcommand{\supp}{\mathrm{supp}\,}
\newcommand{\kk}{^{[k]}}
\newcommand{\sat}{^\mathrm{sat}}

\newtheorem{theorem}{Theorem}[section]
\newtheorem{lemma}[theorem]{Lemma}
\newtheorem{proposition}[theorem]{Proposition}
\newtheorem{cor}[theorem]{Corollary}
\newtheorem{defi}[theorem]{Definition}
\newtheorem{construction}[theorem]{Construction}
\title{\Large \textbf{Positivity of polynomials in the symbolic square}}

\date{}

\begin{document}

\setstretch{1.2}
\maketitle
\section{Introduction}

We denote as $\bb{P}^n$ the $n$-dimensional projective space over $\bb{C}$. We write $S=\bb{C}[x_0,\cdots,x_n]$ and $\m=(x_0,\cdots,x_n)\subset S$.

\begin{defi}
    Let $I\subseteq S$ be a homogeneous ideal. The \textbf{saturation} of $I$ is defined to be
    \begin{equation*}
        I\sat = \{ \; f\in S \mid f\m^k\subseteq I\mathrm{\;for\;some\;}k\geqslant 0 \; \}.
    \end{equation*}
\end{defi}

\begin{theorem}[\cite{SAT}]
    Let $X\subseteq \bb{P}^n$ be a smooth variety and $I\subseteq S$ be its homogeneous ideal. Then $I^{(d)}=(I^d)\sat$ for all $d$.
\end{theorem}

\begin{theorem}\thlabel{sat}
    Let $X\subseteq \bb{P}^n$ be a smooth irreducible variety and $I\subseteq S$ be its homogeneous ideal. Let $P\in \bb{P}^n$ be a real point of $X$, then $I^2$ and $I^{(2)}$ coincide when localized at $P$.
\end{theorem}

\begin{proof}
    We only need to prove $I^{(2)}\subseteq I$ when localized at $P$. Let $I=(g_1,\cdots,g_m)$ and $d$ be the maximal degree of the $g_i$. As $I$ is prime, $I^2$ and $I^{(2)}$ agree when localized at $I$. Take any form $f\in I^{(2)}$, then by \thref{sat} we have $f\in (I^2)\sat$, and so $f\m^k\subset I^2$ for some $k$. In particular, $(x_0^k+\cdots+x_n^k)f\in I^2$ for some $k$. Take $k$ to be even, then $x_0^k+\cdots+x_n^k\neq 0$ at $P$, so $f\in I^{(2)}$ when localized at $P$.
\end{proof}

\begin{theorem}
    Let $X\subseteq \bb{P}^n$ be a smooth irreducible variety and $I\subseteq S$ be its homogeneous ideal.
\end{theorem}



\newpage
\bibliographystyle{plain}
\bibliography{reference}
\end{document}
